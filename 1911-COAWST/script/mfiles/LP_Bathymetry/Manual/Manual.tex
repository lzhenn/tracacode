\documentclass[12pt]{article}
\usepackage{amsfonts, amsmath, latexsym, epsfig}
\usepackage{epsf}
\usepackage{url}
\title{Manual of the matlab scripts of {\tt LP Bathymetry}}
\topmargin=-0.5truein
\oddsidemargin=0.25truein
\evensidemargin=0.25truein
\textwidth=6truein
\textheight=9truein
\urlstyle{sf}

%\usepackage{vmargin}
%\setpapersize{custom}{21cm}{29.7cm}
%\setmarginsrb{1.7cm}{1cm}{1.7cm}{3.5cm}{0pt}{0pt}{0pt}{0pt}
%marge gauche, marge haut, marge droite, marge bas.

\author{Mathieu Dutout Sikiri\'c}



\begin{document}
\newcommand{\RR}{\ensuremath{\mathbb{R}}}
\newcommand{\NN}{\ensuremath{\mathbb{N}}}
\newcommand{\QQ}{\ensuremath{\mathbb{Q}}}
\newcommand{\CC}{\ensuremath{\mathbb{C}}}
\newcommand{\ZZ}{\ensuremath{\mathbb{Z}}}
\newcommand{\TT}{\ensuremath{\mathbb{T}}}
\newtheorem{proposition}{Proposition}
\newtheorem{theorem}{Theorem}
\newtheorem{corollary}{Corollary}
\newtheorem{lemma}{Lemma}
\newtheorem{problem}{Problem}
\newtheorem{conjecture}{Conjecture}
\newtheorem{claim}{Claim}
\newtheorem{remark}{Remark}
\newtheorem{definition}{Definition}
\newcommand{\qed}{\hfill $\Box$ }
\newcommand{\proof}{\noindent{\bf Proof.}\ \ }



\maketitle


When one uses the ROMS model, one needs to smooth the bathymetry in
order to get realistic results.
Two roughness factors are involved: the $rx_0$ factor of 
Beckman and Haidvogel:
\begin{equation*}
rx_0=\max_{e\equiv e'} \frac{|h(e) - h(e')|}{h(e)+h(e')}
\end{equation*}
which should not go above $0.2$ and the $rx_1$ factor of Haney which
should not be above $6$ (\cite{haney}).
(both $rx_0$ and $rx_1$ are shown up at the beginning of a ROMS run).

The original physical bathymetry as computed by interpolation and 
sampling is often too rough for the models and a smoothing operation
is needed.
The programs exposed here try given a roughness factor to find the
bathymetry that is nearest to the real one. 
More details are given in \cite{lpmeth}.

%The program exposed here is based on linear programming and 
%given a bathymetry $h^{real}$ and a fixed $r$,
%it finds the unique bathymetry $h^{filt}$ satisfying to
%\begin{equation*}
%\max_{e\equiv e'} \frac{|h^{filt}_e - h^{filt}_{e'}|}{h^{filt}_e+h^{filt}_{e'}}\leq r
%\end{equation*}
%and minimizing $\sum_e |h^{real}(e) - h^{filt}(e)|$.


\section{Availability}

The source of the program is available from \url{http://www.liga.ens.fr/~dutour/Bathymetry/index.html}

The linear programs are solved by the program {\tt lpsolve} (see \cite{lpsolve} for the installation). Note that we do not use the mex facility but the standalone program.
The scripts are {\tt matlab$^{\copyright}$} scripts and so you need to have {\tt matlab$^{\copyright}$} installed.



\section{How to use it}

First of all, you need your bathymetry in the form of an array of the
form {\tt Hobs(eta\_rho, xi\_rho)} and a mask {\tt MSK(eta\_rho, xi\_rho)}.





\subsection{Using {\tt GRID\_LinProgHeuristic}}

The command to do the filtering is then
\begin{verbatim}
>> Hfielt=GRID_LinProgHeuristic(MSK, Hobs, rx0max);
\end{verbatim}
with
\begin{enumerate}
\item {\bf MSK(eta\_rho,xi\_rho)} the mask.
\item {\bf Hobs(eta\_rho,xi\_rho)} the bathymetry.
\item {\bf rx0max} the chosen maximal $rx_0$ factor.
\end{enumerate}
The program uses a divide and conqueer strategy for reducing the time of
the run, that is it uses as subroutine {\bf GRID\_LinearProgrammingSmoothing\_rx0\_simple}, which may be used separately if desired.
If some additional constraint are needed, have a look at 
{\bf GRID\_LinearProgrammingSmoothing\_rx0}.



%\subsection{Using {\tt GRID\_LinearProgrammingSmoothing\_rx0}}
%
%Suppose that you want at some point the bathymetry to only increase or
%at other point to only decrease.
%What you need to do is specify the array {\tt SignConst(eta\_rho, xi\_rho)}
%with
%
%Also, you may want to specify the maximum value of the perturbation
%at a given point. What you need to do is specify the array {\tt AmpConst(eta\_rho, xi\_rho)} with {\tt a=AmpConst(iEta, iXi)} being a constraint such that
%{\tt |Hfilt(iEta,iXi)-Hobs(iEta, iXi)| $\leq$ a $\times$ Hobs(iEta, iXi)}.
%
%
%The command to do the filtering is then
%\begin{verbatim}
%>> Hfilt=GRID_LinearProgrammingSmoothing_rx0(...
%     MSK, Hobs, rx0max, SignConst, AmpConst);
%>> Hfilt=GRID_LinearProgrammingSmoothing_rx0_simple(MSK, Hobs, rx0max);
%\end{verbatim}
%with
%\begin{enumerate}
%\item {\bf MSK(eta\_rho,xi\_rho)} the mask.
%\item {\bf Hobs(eta\_rho,xi\_rho)} the bathymetry.
%\item {\bf rx0max} the chosen maximal $rx_0$ factor.
%\item {\bf SignConst(eta\_rho,xi\_rho)} the sign matrix
%\begin{itemize}
%\item {\tt SignConst(iEta, iXi)=1} if only bathymetry increase are allowed
%at the point {\tt (iEta,iXi)}.
%\item {\tt SignConst(iEta, iXi)=-1} if only bathymetry decrease are allowed
%at the point {\tt (iEta,iXi)}.
%\item {\tt SignConst(iEta, iXi)=0} if bathymetry is allowed to decrease and
%increase at the point {\tt (iEta,iXi)}.
%\end{itemize}
%\item {\bf AmpConst(eta\_rho,xi\_rho)} for the maximal amplitude of perturbation.\\
%We will have the relation
%\begin{equation*}
%|Hfilt(iEta,iXi)-Hobs(iEta, iXi)|\leq AmpConst(iEta,iXi) \times Hobs(iEta, iXi).
%\end{equation*}
%
%\end{enumerate}




\subsection{Using {\tt GRID\_LinearProgrammingSmoothing\_rx0\_volume}}

If you want to preserve the total volume, then a variation of the
above is:
\begin{verbatim}
>> Hfilt=GRID_LinearProgrammingSmoothing_rx0_volume(MSK, Hobs, rx0max, AreaMatrix);
\end{verbatim}
with
\begin{enumerate}
\item {\bf MSK(eta\_rho,xi\_rho)} the mask of the grid
\item {\bf Hobs(eta\_rho,xi\_rho)} the observed bathymetry of the grid
\item {\bf AreaMatrix(eta\_rho,xi\_rho)} the areas of the wet and land $\rho$-points of the grid.
\item {\bf rx0max, rx1max} are roughness factors.
\end{enumerate}






\subsection{Using {\tt GRID\_LinProgSmoothVertVert\_rx0}}

Sometimes, you want to smooth the bathymetry but preserve the
total volume. 
Here the method is significantly different: We increase the bathymetry
at one cell $e$ by say, $\delta_{e,e'}$ and decrease it at an adjacent
cell $e'$ by $\delta_{e,e'}$.
We minimize the quantity
\begin{equation*}
\sum_{e\equiv e'} |\delta_{e,e'}|
\end{equation*}
This method obviously preserve the volume and tend to preserve the volume
of structures like basin and seamounts.

This method is used in the following way.
\begin{verbatim}
>> Hfilt=GRID_LinProgSmoothVertVert_rx0(MSK, Hobs, r);
\end{verbatim}
with $r$ the roughness factor you want to achieve.
The problem of this method is its high computational cost since
the number variable is higher.




\subsection{Using {\tt GRID\_LinProgHeuristic\_rx0\_fixed}}

This command corrects the bathymetry (if possible) and leaves the bathymetry of a set of points invariant
\begin{verbatim}
>> Hfilt=GRID_LinProgHeuristic_rx0_fixed(MSK, Hobs, PRS, r);
Hfilt=
\end{verbatim}
with
\begin{enumerate}
\item {\bf MSK(eta\_rho,xi\_rho)} the mask of the grid.
\item {\bf Hobs(eta\_rho,xi\_rho)} the original bathymetry of the grid.
\item {\bf PRS(eta\_rho,xi\_rho)} the list of grid point for which
we want to preserve the bathymetry
(PRS(iEta, iXi) == 1 if we want to preserve it).
\item {\bf rx0max} is the maximum $rx_0$ factor.
\end{enumerate}
The program uses a divide and conqueer strategy for reducing the time of
the run, that is it uses as subroutine {\bf GRID\_LinearProgrammingSmoothing\_rx0\_fixed}, which may be used separately if desired.


%\subsection{Using {\tt GRID\_LinearProgrammingSmoothing\_rx0\_fixed}}
%This command uses linear programming for smoothing the bathymetry
%but leaves a set of cells unchanged, which can be useful if for example
%the boundary is to be preserved.
%\begin{verbatim}
%>> Hfilt=GRID_LinearProgrammingSmoothing_rx0_fixed(MSK, Hobs, PRS, rx0max);
%\end{verbatim}
%with
%\begin{enumerate}
%\item {\bf MSK(eta\_rho,xi\_rho)} the mask of the grid
%\item {\bf Hobs(eta\_rho,xi\_rho)} the observed bathymetry of the grid
%\item {\bf PRS(eta\_rho,xi\_rho)} the set of points preserved $1$ for preserved, $0$ for movable.
%\item {\bf rx0max} the maximal roughness factor.
%\end{enumerate}





%\subsection{Using {\tt GRID\_LinProgHeuristic}}
%
%This command uses a splitting procedure to dramatically speed up the
%smoothing.
%\begin{verbatim}
%>> Hfilt=GRID_LinProgHeuristic(MSK, Hobs, r);
%\end{verbatim}
%it should return the same result as {\rm GRID\_LinearProgrammingSmoothing\_rx0\_simple} but much faster.







\subsection{Using {\tt GRID\_LinearProgrammingSmoothing\_rx0\_blockconstraint}}

This command corrects the bathymetry (if possible) and returns a bathymetry satisfying a number of block condition:
\begin{verbatim}
>> Hfilt=GRID_LinearProgrammingSmoothing_rx0_blockconstraint(...
     MSK, Hobs, r, ListVal, ListBlock);
\end{verbatim}
with
\begin{enumerate}
\item {\bf MSK(eta\_rho, xi\_rho)} the mask of the grid.
\item {\bf Hobs(eta\_rho, xi\_rho)} the original bathymetry of the grid.
\item {\bf ListVal(nbBlock,1)} the list of values of constraints.
\item {\bf ListBlock(nbBlock,eta\_rho,xi\_rho)} the list of arrays of constraints. We should have for all $1\leq i\leq nbBlock$ the constraints
\begin{equation*}
\sum_{iEta, iXi} ListBlock(iEta, iXi) (h(iEta, iXi)- h^{obs}(iEta, iXi)) \leq ListVal(i,1)
\end{equation*}


\end{enumerate}


\subsection{Using {\tt GRID\_SmoothPositive\_*}}
This command makes the bathymetry correct by increasing it.
\begin{verbatim}
>> Hfilt=GRID_SmoothPositive_rx0(MSK, Hobs, rx0max);
>> Hfilt=GRID_SmoothPositive_ROMS_rx1(...
     MSK, Hobs, rx1max, ARVD);
\end{verbatim}
with
\begin{enumerate}
\item {\bf MSK(eta\_rho,xi\_rho)} the mask of the grid
\item {\bf Hobs(eta\_rho,xi\_rho)} the observed bathymetry of the grid
\item {\bf rx0max, rx1max} are roughness factors.
\item {\bf ARVD} is the record of vertical parameterization the $S$-coordinates parameters.
\begin{verbatim}
  ARVD.Vtransform=2;
  ARVD.Vstretching=1;
  ARVD.ThetaS=4; % named THETA_S in the roms.in file
  ARVD.ThetaB=0.35; % named THETA_B in the roms.in file
  ARVD.hc=10; % named TCLINE in the roms.in file
  ARVD.N=30;
\end{verbatim}



\end{enumerate}





%\subsection{Using {\tt GRID\_SmoothNegative\_*}}
%This command makes the bathymetry correct by lowering it.
%\begin{verbatim}
%>> Hfilt=GRID_SmoothNegative_rx0(MSK, Hobs, rx0max);
%>> Hfilt=GRID_SmoothNegative_rx1(...
%     MSK, Hobs, rx1max, Scoord, Ccoord, hc);
%>> Hfilt=GRID_SmoothNegative_ROMS_rx1(...
%     MSK, Hobs, rx1max, ThetaS, ThetaB, N, hc);
%\end{verbatim}
%with
%\begin{enumerate}
%\item {\bf MSK(eta\_rho,xi\_rho)} the mask of the grid
%\item {\bf Hobs(eta\_rho,xi\_rho)} the observed bathymetry of the grid
%\item {\bf rx0max, rx1max} are roughness factors.
%\item {\bf Scoord, Ccoord, hc} are the $S$-coordinates parameters.
%\item {\bf ThetaS, ThetaB, N, hc} are ROMS vertical parameters.
%\end{enumerate}






\subsection{Using {\tt GRID\_PlusMinusScheme\_rx0}}

This command makes the bathymetry correct by doing a sequence
of increase/decrease at adjacent cells (see \cite{conundrum}).
\begin{verbatim}
>> [RetBathy, HmodifVal]=GRID_PlusMinusScheme_rx0(...
       MSK, Hobs, rx0max, AreaMatrix);
\end{verbatim}
with
\begin{enumerate}
\item {\bf MSK(eta\_rho,xi\_rho)} the mask of the grid
\item {\bf Hobs(eta\_rho,xi\_rho)} the observed bathymetry of the grid
\item {\bf AreaMatrix(eta\_rho,xi\_rho)} the areas of the wet and dry $\rho$-points.
\item {\bf rx0max, rx1max} are roughness factors.
\end{enumerate}



\subsection{Using {\tt GRID\_LaplacianSelectSmooth\_rx0}}

This command makes the bathymetry correct by doing an iterated
sequence of laplacian filterings
\begin{verbatim}
>> Hfilt=GRID_LaplacianSelectSmooth_rx0(MSK, Hobs, rx0max);
\end{verbatim}
with
\begin{enumerate}
\item {\bf MSK(eta\_rho,xi\_rho)} the mask of the grid
\item {\bf Hobs(eta\_rho,xi\_rho)} the observed bathymetry of the grid
\item {\bf rx0max} the maximal roughness factor.
\end{enumerate}



\section{Notes and Recommendations}

\begin{itemize}
\item The smoothing with respect to $rx_0$ is best done with 
{\bf GRID\_LinProgHeuristic} which uses a linear programming approach
and should be fast even in very large and not pathological grids.
\item The smoothing with respect to $rx_1$ is problematic since
the number of constraint is much larger. Also for {\tt Vtransform=2}
those constraints are nonlinear. A variant of the Martinho Batteen 
\cite{martinho} is
implemented in {\tt GRID\_SmoothPositive\_ROMS\_rx1} and deals with
all the vertical parametrization available in ROMS.
\item The function {\tt GRID\_LaplacianSelectSmooth} has several
advantages over the function {\tt smth\_bath.m} of the ROMS matlab
package:
\begin{itemize}
\item It respects the mask
\item It is guaranteed to terminate
\item It creates a perturbation to the bathymetry of smaller amplitude.
\end{itemize}
Still our recommendation is not to use Laplacian/Shapiro filtering
as they produce worse solution than other methods and have a very
tenuous justification as an adequate method.
\item If preserving the volume is important, you can use 
the function {\tt GRID\_PlusMinusScheme\_rx0}. It will always produce
a larger perturbation than {\tt GRID\_LinearProgrammingSmoothing\_rx0\_volume}
or {\tt GRID\_LinProgSmoothVertVert} but it is much faster.
%Its advantage however is in speed.
%\item If preserving volume is important, then you have two options:
%\begin{enumerate}
%\item Use {\tt GRID\_LinearProgrammingSmoothing\_volume}, which has
%a global preservation of the volume.
%\item Use {\tt GRID\_LinProgSmoothVertVert} or its variant 
%{\tt GRID\_PlusMinusScheme}, which preserve the volume locally.
%\end{enumerate}
%\item If there is some specific feature you want to preserve or some
%more specific request, then you may consider using 
%{\tt GRID\_LinearProgrammingSmoothing}, but you have to be careful
%with your requirement as there may be no solution at all.
\end{itemize}




%\section{How does it work}
%
%Given $m$ vectors $v_i$ in $\RR^n$, $m$ reals $b_i$ and a vector $w$,
%a linear program is the following problem:
%\begin{equation*}
%\begin{array}{rcl}
%\mbox{{\bf maximize}} &\mbox{~~}& w^T x\\
%\mbox{{\bf subject to}} &\mbox{~~}& v_i^T x\geq b_i
%\end{array}
%\end{equation*}
%This kind of problems (\cite{chvatal}) have an extended history and theory.
%They have a unique solution and methods exist for solving them
%when $n$ and $m$ are of the order of million of variables.
%The game is then to transform the minimization problem,
%which is apparently not linear into one which is linear by adding some
%new variables. We used the {\tt lpsolve} program because it has a
%support for sparse matrices.



%\section{Last remarks}
%The computational time is roughly 5 minutes with a $60\times 160$ grid
%and it can reach several hours for a $160\times 360$ grid.
%This is an important cost but only a one time cost, since the grid is
%correct once and for all.
%If you have any questions, please drop me a line
%to {\tt mdsikir@irb.hr} and I will reply
%as much as I can.





\begin{thebibliography}{WWw99}
\bibitem{haney}
R.L. Haney, {\em On the pressure gradient force over steep bathymetry in sigma coordinates ocean models}, Journal of Physical Oceanography {\bf 21} (1991) 610--619.

%\bibitem[MeBl85]{mellor}
%G.L. Mellor and A.F. Blumberg, {\em Modeling vertical horizontal diffusivities with the sigma coordinate system}, Monthly Weather Review {\bf 113} (1985) 1379--1383.

\bibitem{lpmeth}
M. Dutour Sikiric, I. Janekovic, M. Kuzmic, {\em A new approach to bathymetry smoothing in sigma-coordinate ocean models}, 
Ocean Modelling, Volume 29, Issue 2, 2009, Pages 128-136

\bibitem{chvatal}
V.~Chv{\'a}tal, {\em Linear Programming}, W.H.~Freeman and Company, 1983.

\bibitem{conundrum}
G.L. Mellor, T. Ezer and L.-Y. Oey, {\em The pressure gradient conundrum of Sigma coordinate Ocean models}, Journal of atmospheric and oceanic technology {\bf 11} (1994) 1126--1134.

\bibitem{martinho}
A.S. Martinho and M.L. Batteen, {\em On reducing the slope parameter in terrain following numerical ocean models}, Ocean Modelling {\bf 13} (2006) 166--175.


\bibitem{lpsolve}
P. Notebaert and K. Eikland, \url{http://lpsolve.sourceforge.net/5.5/}

\end{thebibliography}
\end{document}
